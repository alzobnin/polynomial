\documentclass[a4paper,reqno,12pt]{amsart}
\usepackage[T2A]{fontenc}
\usepackage[utf8]{inputenc}
\usepackage[russian]{babel}
\usepackage{amsfonts, amssymb, amsmath, amsthm}

\usepackage{geometry}
\geometry{a4paper,top=2cm,bottom=2cm,left=2cm,right=2cm}

\righthyphenmin=2
\sloppy

\theoremstyle{plain}
\newtheorem{theorem}{Теорема}[section]
\newtheorem{proposition}[theorem]{Предложение}
\newtheorem{corollary}[theorem]{Следствие}
\newtheorem{lemma}[theorem]{Лемма}

\theoremstyle{remark}
\newtheorem*{remark}{Замечание}

\theoremstyle{definition}
\newtheorem{definition}[theorem]{Определение}
\newtheorem{example}{Пример}[section]
\newtheorem*{examples}{Примеры}
\newtheorem{problem}{Задача}[section]

\renewcommand{\proofname}{Доказательство}
\renewcommand{\refname}{Список литературы}

\renewcommand{\le}{\leqslant}
\renewcommand{\ge}{\geqslant}
\renewcommand{\epsilon}{\varepsilon}
\renewcommand{\phi}{\varphi}
\renewcommand{\kappa}{\varkappa}
\DeclareMathOperator{\lcm}{\text{НОК}}
\DeclareMathOperator{\mygcd}{\text{НОД}}
\DeclareMathOperator{\wt}{wt}
\DeclareMathOperator{\lt}{LT}
\DeclareMathOperator{\lm}{LM}
\DeclareMathOperator{\lp}{lp}
\DeclareMathOperator{\lc}{lcf}
\DeclareMathOperator{\lex}{lex}
\DeclareMathOperator{\deglex}{deglex}
\DeclareMathOperator{\degrevlex}{degrevlex}
\DeclareMathOperator{\Ker}{Ker}
\DeclareMathOperator{\ord}{ord}
\newcommand{\Mon}{\mathbb{M}}
\newcommand{\M} {\mathbb{M}}
\newcommand{\N}{\mathbb{N}}
\newcommand{\Z}{\mathbb{Z}}
\newcommand{\Q}{\mathbb{Q}}
\newcommand{\R}{\mathbb{R}}
\renewcommand{\C}{\mathbb{C}}
\newcommand{\bs}{\boldsymbol}
\newcommand{\Mat}{\mathcal{M}}
\newcommand{\admOrd}{\prec}
\newcommand{\admOrdLe}{\preccurlyeq}
\newcommand{\admOrdG}{\succ}
\newcommand{\admOrdGe}{\succcurlyeq}
\newcommand{\Zplusn}{\Z_{\ge 0}^n}

\begin{document}

\section*{Введение}

Кольцо многочленов $F[x]$ над полем $F$ от одной переменной $x$ обладает рядом хороших свойств.
В нём есть алгоритм деления с остатком и алгоритм Евклида. Все идеалы в этом кольце главные. В этом смысле оно очень похоже на кольцо целых чисел $\Z$.
Задача принадлежности многочлена $f$ идеалу $I$ в таком кольце решается тривиально с помощью деления с остатком на образующий элемент идеала $I$. Вычисления в факторкольце приводят к арифметике остатков по модулю образующего многочлена.

Если переменных становится хотя бы две, то таких хороших свойств уже нет. 
Например, непонятно, что значит <<разделить $x$ на $y$ с остатком>>.
Идеалы в кольце $F[x,y]$ уже не обязательно главные.
Системы полиномиальных уравнений от двух и более переменных, вообще говоря, уже не эквивалентны одному уравнению.

Наша ближайшая цель --- научиться исследовать системы нелинейных полиномиальных уравнений над полем.
В частности, мы хотим научиться отвечать на следующие вопросы:
\begin{itemize}
\item[---] всякая ли система может быть задана конечным набором уравнений?
\item[---] как проверить, совместна ли система?
\item[---] как узнать, конечно ли множество решений системы (над $\bar{F}$)?
\item[---] как проверить принадлежность многочлена $f$ идеалу $I$?
\item[---] как проводить вычисления в факторалгебре $F[x_1, \ldots, x_n]/I$?
\item[---] как исключить заданные неизвестные из системы?
\end{itemize}

Заметим, что для линейных систем ответы на эти вопросы легко получить с помощью приведения системы к ступенчатому виду методом Гаусса.


\section{Свойства конечности полугруппы $\Zplusn$}
Мы будем работать в кольце многочленов $F[x_1, \ldots, x_n]$.
Договоримся называть \emph{мономом} выражение $x_1^{a_1} \ldots x_n^{a_n}$, где $a_i$~--- неотрицательные целые числа. \emph{Термом} будем называть одночлен, то есть, моном с коэффициентом.

Моноид мономов изоморфен моноиду $\Zplusn$. Иногда нам будет удобнее говорить о мономах в аддитивном смысле именно как об элементах $\Z_{\ge 0}^n$. Следуя книге~\cite{Zplusn}, рассмотрим <<свойство конечности>> этой полугруппы.

\begin{definition}
\emph{Идеалом} полугруппы $\Zplusn$ называется всякое ее подмножество, содержащее вместе с каждой точкой $\alpha$ и точку $\alpha + \gamma$ $\forall\,\gamma \in \Zplusn$.
\end{definition}

\begin{definition}
\emph{Октантом} $O(\alpha)$ с центром в точке $\alpha$ называется множество $\{\alpha + \gamma \, | \, \gamma \in \Zplusn\}$.
\end{definition}

Ясно, что октант является идеалом, и что идеал вместе с каждой своей точкой содержит весь октант с центром в ней.

\begin{theorem}[Свойство конечности полугруппы $\Zplusn$]
Всякий идеал в $\Zplusn$ является объединением конечного числа октантов.
\end{theorem}

\begin{proof}
\end{proof}

\begin{definition}
Подмножество $\Zplusn$ называется \emph{коидеалом}, если его дополнение~--- идеал.
\end{definition}

\begin{corollary}
\label{HilbertBasisMonomial}
Всякий мономиальный идеал в $F[x_1,\ldots,x_n]$ конечно порожден.
\end{corollary}

\begin{corollary}[Лемма Диксона]
Пусть $M_1, \ldots, M_s, \ldots$~--- бесконечная последовательность мономов. Тогда
обязательно найдутся два монома $M_i$ и $M_j$, такие, что $M_i \mid M_j$.
\end{corollary}

\begin{proposition}
Моном $M$ принадлежит мономиальному идеалу $(M_1, \ldots, M_s)$ тогда и только тогда, когда $M_i \mid M$ для некоторого~$i$.
\end{proposition}


Далее нам понадобится понятие допустимого упорядочения на мономах, чтобы можно было выбирать <<старший моном>> в многочленах.

\begin{definition}
\emph{(Допустимым) мономиальным упорядочением} $\admOrd$ на $\M_n$
называется линейный порядок, удовлетворяющий свойствам:
\begin{enumerate}
 \item $M \admOrd N \; \Longrightarrow M P \admOrd N P \quad \forall \, M,N,P \in \M_n$;
 \item $1 \admOrdLe M \quad \forall \, M \in \M_n$.
\end{enumerate}
\end{definition}

\begin{definition}
 Пусть $a = (a_1, \ldots, a_n), \, b = (b_1, \ldots, b_n) \in \R^n$.
Вектор $a$ \emph{лексикографически младше} вектора $b$ ($a \admOrd_{\lex} b$), если
существует такое $k$, $0 \le k < n$, что $a_i = b_i$ для всех натуральных $i \le k$, но $a_{k+1} < b_{k+1}$.
Аналогично будем сравнивать векторы-столбцы.
\end{definition}

\begin{examples}
 Зафиксируем порядок на переменных (положив, например, $x_1 \admOrd x_2 \admOrd \ldots \admOrd x_n$).
Пусть $M = x_1^{a_1} \ldots x_n^{a_n}$ и $N = x_1^{b_1} \ldots x_n^{b_n}$~--- произвольные
мономы из $\M_n$.
Следующие бинарные отношения на $\M_n$ являются мономиальными упорядочениями:
\begin{enumerate}
 \item \emph {Лексикографическое} упорядочение ($\lex$):
  $$
   M \admOrd_{\lex} N \; \iff \; (a_1, \ldots, a_n) \admOrd_{\lex} (b_1, \ldots, b_n).
  $$
 \item \emph {Сначала по степени, затем лексикографическое} упорядочение ($\deglex$):
  $$
   M \admOrd_{\deglex} N \; \iff \; (\deg M, a_1, \ldots, a_n) \admOrd_{\lex} (\deg N, b_1, \ldots, b_n).
  $$
 \item \emph {Сначала по степени, затем обратное лексикографическое} упорядочение ($\degrevlex$):
  $$
   M \admOrd_{\degrevlex} N \; \iff \; (\deg M, b_n, \ldots, b_1) \admOrd_{\lex} (\deg N, a_n, \ldots, a_1).
  $$
\end{enumerate}
\end{examples}

\begin{proposition}
 Любое мономиальное упорядочение вполне упорядочивает множество $\M_n$
(то есть, любая убывающая цепочка мономов обрывается). \qed
\end{proposition}

Зафиксируем некоторое упорядочение на мономах.
Будем обозначать через $\lm f$ старший моном ненулевого многочлена $f$, а через $\lm I$ множество
старших мономов ненулевых элементов из $I$.

\begin{theorem}
Пусть $I$~--- идеал в $F[x_1, \ldots, x_n]$. Рассмотрим его как векторное пространство над $F$.
Пусть подпространство $L$ порождено всеми мономами, не принадлежащими $\lm I$.
Тогда справедливо разложение векторных пространств
$$
 F[x_1, \ldots, x_n] = I \oplus L.
$$
\end{theorem}
\begin{proof}

\end{proof}


\begin{definition}
Система образующих $G$ идеала $I$ называется его \emph{базисом Грёбнера}, если $(\lm G) = (\lm I)$. 
\end{definition}

Следствие~\ref{HilbertBasisMonomial} показывает, что у каждого идеала существует конечный базис Грёбнера.
Действительно, достаточно взять многочлены $g_1, \ldots, g_s \in I$, такие, 
что $\lm g_1, \ldots, \lm g_s$ порождают мономиальный идеал $(\lm I)$.

\begin{corollary}[Теорема Гильберта о базисе]
Кольцо многочленов $F[x_1, \ldots, x_n]$ нётерово, то есть, всякий идеал в нем конечно порожден.
\end{corollary}


\bigskip


\begin{problem}
Для всякого подмножества $I \subset \{1, \ldots, n\}$ можно рассмотреть подполугруппу 
$$
 \Zplusn(I) := \{(a_1, \ldots, a_n) \in \Zplusn \, | \, a_i = 0 \, \forall\, i \in I \}.
$$
\emph{Сдвинутой координатной подполугруппой} называется множество вида $\alpha + \Zplusn(I)$, где $\alpha \in \Zplusn$.

Докажите, что всякий коидеал в $\Zplusn$ является объединением конечного числа сдвинутых координатных подполугрупп.
\end{problem}


\begin{problem}
Докажите, что из любой бесконечной последовательности мономов можно выбрать подпоследовательность,
в которой каждый следующий моном делится на предыдущий.
\end{problem}

\begin{problem}
Покажите, что в $F[x,y]$ упорядочения $\deglex$ и $\degrevlex$ совпадают.
\end{problem}


\begin{problem}
Покажите, что на множестве мономов из $F[x,y]$ существует континуум различных упорядочений.
\end{problem}

\begin{problem}
Зафиксируем лексикографический порядок с $x \admOrdG y \admOrdG z$.
Верно ли, что множество $\{x - z^2, y - z^3\}$ является базисом Грёбнера идеала, порожденного этими двумя многочленами?
\end{problem}

\begin{problem}
Приведите пример ассоциативного коммутативного кольца с единицей, не являющегося нётеровым.
\end{problem}


\bigskip


\section{Матричное задание мономиальных упорядочений}

Обозначим через $\M_n$~--- множество мономов от $n$ переменных $x_1, \ldots, x_n$.
Пусть дана матрица $\Mat \in M_{m,n} (\R)$ размера ${m \times n}$ (при некотором $m \ge 1$)
с нулевым ядром над $\Q$ и лексикографически положительными столбцами.
Тогда можно задать мономиальное упорядочение на $\M_n$ следующим образом:
$$
 x_1^{\alpha_1} \ldots x_n^{\alpha_n} \:\admOrd\: x_1^{\beta_1} \ldots x_n^{\beta_n} \:
\iff \: \Mat \begin{pmatrix} \alpha_1 \\ \vdots \\ \alpha_n \end{pmatrix} \admOrd_{\lex}
 \Mat \begin{pmatrix} \beta_1 \\ \vdots \\ \beta_n \end{pmatrix}.
$$

Матрицы с указанными свойствами мы будем называть \emph{мономиальными}.
Допуская вольность записи, мы будем писать $\Mat \cdot P$, 
подразумевая здесь под $P$ вектор-столбец из степеней переменных монома~$P$.
Итак, если матрица $\Mat$ задает упорядочение $\admOrd$, то
$$
 P \admOrd Q \; \iff \; \Mat \cdot P \admOrd_{\lex} \Mat \cdot Q.
$$
Произведение первой строки мономиальной матрицы на вектор-столбец степеней монома~$M$
мы будем называть \emph{весом}~$M$ относительно этой строки.

По определению единичная матрица задает лексикографическое упорядочение.
Разумеется, одно и то же упорядочение может быть задано разными матрицами.
Так, можно доказать, что лексикографическое упорядочение
задается произвольными нижнетреугольными матрицами с положительными элементами на диагонали.
Кроме того, можно доказать, что если две мономиальные матрицы задают одно и то же упорядочение и имеют рациональные коэффициенты,
то одна получается из другой умножением слева на нижнетреугольную
матрицу с положительными элементами на диагонали.

\begin{example}
 Упорядочение $\deglex$ с $x_1 \admOrdG x_2 \admOrdG \ldots \admOrdG x_n$ задается матрицей ${n \times n}$
{
$$
 \begin{pmatrix}
  1 & 1 & 1 & \ldots & 1 & 1 & 1 \\
  1 &                            \\    
    & 1                          \\    
    &   & 1                      \\   
    &   &   & \ldots             \\
    &   &   &        & 1         \\
    &   &   &        &   & 1     \\
 \end{pmatrix}.
$$
}
\end{example}

\begin{example}
 Упорядочение $\degrevlex$ с $x_1 \admOrd x_2 \admOrd \ldots \admOrd x_n$ можно задать матрицей ${n \times n}$
{
$$
 \begin{pmatrix}
   1 &  1 &  1 & \ldots &  1 &  1 &  1 \\
     &    &    &        &    &    & -1 \\   
     &    &    &        &    & -1      \\
     &    &    &        & -1           \\
     &    &    & \ldots                \\
     &    & -1                         \\
     & -1                              \\
 \end{pmatrix}.
$$
}
или матрицей
{
$$
 \begin{pmatrix}
   1 &  1 &  1 & \ldots &  1 &  1 &  1 \\
   1 &  1 &  1 & \ldots &  1 &  1      \\   
   1 &  1 &  1 & \ldots &  1           \\
\hdotsfor{7}\\
   1 &  1                              \\
   1                                   \\
 \end{pmatrix}.
$$
}
\end{example}


Мы докажем обратное утверждение:

\begin{theorem}
Каждое мономиальное упорядочение на~$\M_n$ можно задать матрицей $\Mat \in M_{m,n} (\R)$ размера ${m \times n}$
(при некотором $m \ge 1$)
с нулевым ядром над $\Q$ и лексикографически положительными столбцами
\end{theorem}

Доказательство этого факта было известно давно~\cite{Tre}, 
но в контексте базисов Гребнера было впервые опубликовано в 1986 Лоренцо Робьяно~\cite{Robb1, Robb2}.
Робьяно впоследствии развил эти идеи, которые привели к построению так называемых вееров Гребнера (Gr\"obner Fan) и алгоритму маршрута Гребнера (Gr\"obner Walk).
Мы приведем более простое доказательство, принадлежащее Хуну Хонгу и Фолькеру Вайспфеннингу~\cite{HongWeispf}.

\begin{lemma}
 Пусть $\admOrd$~--- допустимый порядок на $\Q^n$ и $U$~--- ненулевое подпространство в евклидовом пространстве $\Q^n$.
Тогда существует единственная строка $A \in \langle U \rangle_{\R}$, такая, что $\| A \| = 1$ и для любого вектора-столбца $x \in U$
неравенство $A \cdot x > 0$ влечет $x \admOrdG 0$.
\end{lemma}

\begin{proof}
Докажем {\bf существование} такой строки $A$.
Пусть $v_1, \ldots, v_s$~--- ортогональный базис пространства $U \subset \Q^n$.
Его всегда можно выбрать с условием $v_i \admOrdG 0$.
Пусть $v_k$~--- максимальный базисный вектор.
Положим
$$
 \beta = \sum_{i=1}^s \gamma_i \frac{v_i^T}{\| v_i \|},
$$
а
$$ 
 \gamma_i = \inf \{ q \in \Q \mid q v_k \admOrdG v_i\}.
$$ 
Заметим, что $\gamma_i \le 1$ и, вообще говоря, $\gamma_i \in \R$.
Положим $A = \frac{\beta}{|\beta|}$.

Пусть $A \cdot x > 0$ для некоторого вектора $x \in U$.
Разложим вектор $x$ по базису $v_1, \ldots, v_s$:
$$
 x = \sum_{i=1}^s x_i' v_i, \quad x_i' \in \Q.
$$
Тогда, ввиду ортогональности базиса $v_1, \ldots, v_n$, получаем
$$
 \beta \cdot x = \sum_{i=1}^s \gamma_i x_i' > 0.
$$
Мы можем выбрать такие рациональные числа $\gamma_i'$,
что по прежнему будет выполняться неравенство 
$$
 \sum_{i=1}^s \gamma_i' x_i' > 0,
$$
причем
\begin{gather*}
 \gamma_i' < \gamma_i \text{ при } x_i' > 0,\\
 \gamma_i' > \gamma_i \text{ при } x_i' < 0 \\
 \text{ и } \gamma_i' = 0 \text{ при } x_i' = 0.
\end{gather*}
Тогда $\gamma_i' x_i' v_k \admOrd \gamma_i x_i' v_k \admOrd x_i' v_i$.
Осталось заметить, что
$$
 0 = 0 v_k \admOrd \left(\sum_{i=1}^s \gamma_i' x_i' \right) v_k \admOrd \sum_{i=1}^s x_i' v_i = x.
$$ 

Теперь докажем {\bf единственность} строки $A$.
Пусть имеются две разные строки $A$ и $A'$ с указанным свойством.
По условию $\| A \| = \| A' \| = 1$.
Тогда множество 
$$
 S = \{ x \in \langle U \rangle_{\R} \,|\, Ax > 0, \, A'x < 0\}
$$
которое есть пересечение двух полупространств в $\langle U \rangle_{\R}$,
является непустым и открытым.
Значит, в нем существует точка $x$ с рациональными координатами.
Тогда $x \in U$, причем $Ax > 0$ и $A'(-x) > 0$,
откуда $x \admOrdG 0$ и $-x \admOrdG 0$.
Складывая два последних неравенства, получаем противоречие: $0 \admOrdG 0$.
\end{proof}

{\it Доказательсто теоремы.}
Будем строить искомую матрицу индуктивно.
Выберем в качестве ее первой строки $A_1$ строку, которая существует по доказанной лемме для пространства $U_0 = \Q^n$.
Пусть построено $k$ строк $A_1, \ldots, A_k$ этой матрицы.
Обозначим за $U_k \subset \Q^n$ подпространство всех рациональных векторов, ортогональных каждой из этих строк,
и построим строку $A_{k+1}$ по лемме для этого подпространства $U_k$.
На некотором шаге $m$ мы получим $U_m = \{0\}$, так как размерности пространств $U_k$ уменьшаются на каждом шаге.
Таким образом, мы построим требуемую матрицу размера $m \times n$.\qed


\medskip

Заметим, что в построенной матрице норма каждой строки равна единице, а сами строки ортогональны друг другу.
Матрица мономиального упорядочения с таким свойством будет единственной (а потому ее можно назвать \emph{канонической матрицей} 
данного мономиального упорядочения). В самом деле, каждая очередная строка этой матрицы по доказанной лемме определяется единственным образом.

\bigskip

Ясно, что существует только одно упорядочение на мономах от одной переменной~--- упорядочение по степени.
Интересен следующий факт:
\begin{proposition}
При $n \ge 2$ существует континуум различных мономиальных упорядочений.
\end{proposition}

\begin{proof}
Мощность множества различных канонических матриц, задающих мономиальные упорядочения, не превосходит мощности континуума.
Поэтому достаточно заметить, что матрицы $\begin{pmatrix} 1 & \alpha \end{pmatrix}$
при различных иррациональных $\alpha$ задают разные упорядочения на мономах от двух переменных.
\end{proof}

\begin{problem}
Постройте канонические матрицы упорядочений $\deglex$ и $\degrevlex$.
\end{problem}


\begin{thebibliography}{99}
\bibitem{Zplusn} А. Г. Хованский, С. П. Чулков.
Геометрия полугруппы $\Zplusn$: приложения к комбинаторике, алгебре и дифференциальным уравнениям.
М., МЦНМО, 2006.

\bibitem{Cox}
 Cox D., Little J., O'Shea D.,
 {\it Ideals, Varieties and Algorithms.
  An Introduction to Computational Algebraic Geometry and Commutative Algebra,}
 New York, NY: Springer, 1998.
  [Имеется перевод: Кокс Д., Литтл Дж., О'Ши Д.,
   {\it Идеалы, многообразия и алгоритмы,}
    М., Мир, 2000.]

\bibitem{HongWeispf}
 Hong H. and Weispfenning V.,
 {\it Algorithmic Theory of Admissible Term Orders,}
 preprint, 1999.

\bibitem{Robb1}
 Robbiano L.,
 {\it Term Orderings on the Polynomial Ring,}
 in Proceedings of EUROCAL 85,
 Springer Lecture Notes in Computer Science 204, 513-517, 1985.

\bibitem{Robb2}
 Robbiano L.,
 {\it On the Theory of Graded Structures,}
 The Journal of Symbolic Computation, 2, 139-170, 1986.

\bibitem{Tre}
 Trevisan G.,
 {\it Classificazione dei semplici ordinamenti di
  un gruppo libero commutativo con N generatori,}
 Rend. Sem. Mat. Padova, 22, 143-156, 1953.

\end{thebibliography}

\end{document}
